\documentclass{beamer}

\usepackage[utf8]{inputenc}
\usepackage{default}

\mode<presentation>
%{ \usetheme{boxes} }


\usetheme{Madrid}

\usepackage{times}
\usepackage{graphicx}
\usepackage{tabulary}
\usepackage{listings}
\usepackage{verbatimbox}
\usepackage{graphicx}
\usepackage{lmodern}
\usepackage[absolute,overlay]{textpos}
\usepackage{pgfpages}
\usepackage{color}

\pgfdeclareimage[height=1.0cm]{logo_rcc}{../icons/logo_rcc.png}
\setlength{\TPHorizModule}{1mm}
\setlength{\TPVertModule}{1mm}
\newcommand{\RCCLogo}{
\begin{textblock}{14}(1.5,1.5)
  \pgfuseimage{logo_rcc}
\end{textblock}
}

\definecolor{mycolorcli}{RGB}{53,154,26}
\definecolor{mycolorcode}{RGB}{0,0,255}
\definecolor{mycolordef}{RGB}{255,0,0}
\definecolor{mycolorlink}{RGB}{184,4,255}

\title{\huge{Hive}}
\author{Igor Yakushin \\ \texttt{ivy2@uchicago.edu}}
%\date{January 20, 2018}

\definecolor{ChicagoMaroon}{RGB}{128,0,0}

\setbeamercolor{title}{bg=ChicagoMaroon}

\begin{document}

\setbeamertemplate{navigation symbols}{}

\setbeamercolor{fcolor}{fg=white,bg=ChicagoMaroon}
\setbeamertemplate{footline}{
\begin{beamercolorbox}[ht=4ex,leftskip=1.4cm,rightskip=.3cm]{fcolor}
\hrule
\vspace{0.1cm}
   \hfill \insertshortdate \hfill \insertframenumber/\inserttotalframenumber
\end{beamercolorbox}
}

\setbeamercolor{frametitle}{bg=ChicagoMaroon,fg=white}

\begin{frame}
\titlepage
\end{frame}


\section{Hive}
\subsection{Introduction}
\begin{frame}
 \frametitle{Hive: introduction}
 \begin{itemize}
  \item Hive provides Hadoop with SQL access to data
  \item Hive server, sitting on top of HDFS and MapReduce, accepts connections from various Hive clients via, for example, ODBC, JDBC drivers and converts SQL into MapReduce jobs
  \item Hive supports a large subset of SQL including joins
  \item One can create indexes to improve performance
  \item When creating a table, one specifies where the data is stored: textfile, HBase table or any other of numerous data formats supported by Hadoop and residing on HDFS
 \end{itemize}

\end{frame}

\subsection{Example}
\begin{frame}[fragile]
  \frametitle{Hive: example}
{\color{mycolorcode}
  \begin{lstlisting}[frame=single, basicstyle=\tiny]
$ hive
hive> set hive.cli.print.current.db=true;
hive> CREATE DATABASE my1;
hive> USE my1;
hive (my1)> CREATE TABLE t1(W STRING, N INT) ROW FORMAT DELIMITED FIELDS TERMINATED BY ',';
hive (my1)> LOAD DATA INPATH 'words.csv' INTO TABLE t1;
hive (my1)> select count(*) from t1;
hive (my1)> select max(N) from t1;
\end{lstlisting}
}
\begin{itemize}
\item To avoid collision, everybody should use a separate database named by username
\item Interactive usage is only good for experimentation, don't use it for homework. Instead, prepare
  a sql script, for example, {\color{mycolorcli}\verb|my.sql|}, and execute it as follows:
{\color{mycolorcli}
\begin{verbatim}
hive -f my.sql > out.txt 2> err.txt
\end{verbatim}
}
and submit three files as homework: {\color{mycolorcli}\verb|my.sql|}, {\color{mycolorcli}\verb|out.txt|} and 
{\color{mycolorcli}\verb|err.txt|}.
\item You can also run sql on hive database as follows:
{\color{mycolorcli}
\begin{verbatim}
hive --database my1 -e 'select count(*) from t1;'
\end{verbatim}
}
\end{itemize}

\end{frame}

\end{document}
