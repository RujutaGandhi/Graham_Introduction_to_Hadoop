\documentclass{beamer}

\usepackage[utf8]{inputenc}
\usepackage{default}

\mode<presentation>
%{ \usetheme{boxes} }


\usetheme{Madrid}

\usepackage{times}
\usepackage{graphicx}
\usepackage{tabulary}
\usepackage{listings}
\usepackage{verbatimbox}
\usepackage{graphicx}
\usepackage{lmodern}
\usepackage[absolute,overlay]{textpos}
\usepackage{pgfpages}
\usepackage{color}

\pgfdeclareimage[height=1.0cm]{logo_rcc}{../icons/logo_rcc.png}
\setlength{\TPHorizModule}{1mm}
\setlength{\TPVertModule}{1mm}
\newcommand{\RCCLogo}{
\begin{textblock}{14}(1.5,1.5)
  \pgfuseimage{logo_rcc}
\end{textblock}
}

\definecolor{mycolorcli}{RGB}{53,154,26}
\definecolor{mycolorcode}{RGB}{0,0,255}
\definecolor{mycolordef}{RGB}{255,0,0}
\definecolor{mycolorlink}{RGB}{184,4,255}

\title{\huge{Pig}}
\author{Igor Yakushin \\ \texttt{ivy2@uchicago.edu}}
%\date{January 20, 2018}

\definecolor{ChicagoMaroon}{RGB}{128,0,0}

\setbeamercolor{title}{bg=ChicagoMaroon}

\begin{document}

\setbeamertemplate{navigation symbols}{}

\setbeamercolor{fcolor}{fg=white,bg=ChicagoMaroon}
\setbeamertemplate{footline}{
\begin{beamercolorbox}[ht=4ex,leftskip=1.4cm,rightskip=.3cm]{fcolor}
\hrule
\vspace{0.1cm}
   \hfill \insertshortdate \hfill \insertframenumber/\inserttotalframenumber
\end{beamercolorbox}
}

\setbeamercolor{frametitle}{bg=ChicagoMaroon,fg=white}

\begin{frame}
\titlepage
\end{frame}


\section{Pig}
\subsection{Introduction}
\begin{frame}[fragile]
 \frametitle{Pig: introduction}
 \begin{itemize}
  \item High level language - {\color{mycolordef}\textbf{Pig Latin}}
  \item Compiler translates Pig Latin into MapReduce jobs
  \item It is a {\color{mycolordef}\textbf{dataflow language}} where you define a data stream and a set of transformations applied to it.
  \item Operations: load, store, dump, filter, foreach, group, join, order by, distinct, limit, sample, etc.
  \item You can specify data types to help Pig to optimize a program or you can let it figure it out
 \end{itemize}
\end{frame}


\subsection{How to run}
\begin{frame}[fragile]
 \frametitle{Pig: how to run}
 Pig programs can run in three ways: 
 \begin{itemize}
   \item as a script
     \begin{itemize} 
     \item on a local computer
{\color{mycolorcli}
       \begin{lstlisting}[frame=no, basicstyle=\tiny]
         pig -x local milesPerCarrier.pig
       \end{lstlisting}
}
     \item on a cluster
{\color{mycolorcli}
       \begin{lstlisting}[frame=no, basicstyle=\tiny]
         pig -x mapreduce milesPerCarrier.pig
       \end{lstlisting}
}       
     \end{itemize}
   \item interactively using Grunt interpreter
{\color{mycolorcli}
       \begin{lstlisting}[frame=no, basicstyle=\tiny]
         pig -x local
       \end{lstlisting}
}            
   \item Embedded in other languages such as Java, Python, JavaScript
 \end{itemize}
\end{frame}


\subsection{Examples}
\begin{frame}[fragile]
 \frametitle{Pig: examples}
{\color{mycolorcode}
  \begin{lstlisting}[frame=single, basicstyle=\tiny]
records = LOAD 'pig/words.csv' USING PigStorage(',') AS (W, N:int);
mrecs = GROUP records ALL;
tot = FOREACH mrecs GENERATE SUM(records.N);
DUMP tot;
  \end{lstlisting}

  \begin{lstlisting}[frame=single, basicstyle=\tiny]
in = load 'pig/mary.txt' as (line);
-- TOKENIZE splits the line into a field for each word.
-- flatten will take the collection of records returned by
-- TOKENIZE and produce a separate record for each one, calling the single
-- field in the record word.
words = foreach in generate flatten(TOKENIZE(line)) as word;
grpd = group words by word;
cntd = foreach grpd generate group, COUNT(words);
store cntd into 'cntd.out';
  \end{lstlisting}

  \begin{lstlisting}[frame=single, basicstyle=\tiny]
records = LOAD 'words.csv' USING PigStorage(',') AS (W, N:int);
r1 = filter records by N < 10;
dump r1;
r2 = foreach r1 generate N*N as N2, W;
describe r2;
r3 = join r1 by W, r2 by W;
  \end{lstlisting}
}


\end{frame}


\end{document}
