\documentclass{beamer}

\usepackage[utf8]{inputenc}
\usepackage{default}

\mode<presentation>
%{ \usetheme{boxes} }


\usetheme{Madrid}

\usepackage{times}
\usepackage{graphicx}
\usepackage{tabulary}
\usepackage{listings}
\usepackage{verbatimbox}
\usepackage{graphicx}
\usepackage{lmodern}
\usepackage[absolute,overlay]{textpos}
\usepackage{pgfpages}
\usepackage{color}

\pgfdeclareimage[height=1.0cm]{logo_rcc}{../icons/logo_rcc.png}
\setlength{\TPHorizModule}{1mm}
\setlength{\TPVertModule}{1mm}
\newcommand{\RCCLogo}{
\begin{textblock}{14} (1.5,1.5)
  \pgfuseimage{logo_rcc}
\end{textblock}
}

\definecolor{mycolorcli}{RGB}{53,154,26}
\definecolor{mycolorcode}{RGB}{0,0,255}
\definecolor{mycolordef}{RGB}{255,0,0}
\definecolor{mycolorlink}{RGB}{184,4,255}

\title{\huge{How to connect to Hadoop cluster via ssh}}
\author{Igor Yakushin \\ \texttt{ivy2@uchicago.edu}}
%\date{January 20, 2018}

\definecolor{ChicagoMaroon}{RGB}{128,0,0}

\setbeamercolor{title}{bg=ChicagoMaroon}

\begin{document}

\setbeamertemplate{navigation symbols}{}

\setbeamercolor{fcolor}{fg=white,bg=ChicagoMaroon}
\setbeamertemplate{footline}{
\begin{beamercolorbox}[ht=4ex,leftskip=1.4cm,rightskip=.3cm]{fcolor}
\hrule
\vspace{0.1cm}
   \hfill \insertshortdate \hfill \insertframenumber/\inserttotalframenumber
\end{beamercolorbox}
}

\setbeamercolor{frametitle}{bg=ChicagoMaroon,fg=white}

\begin{frame}
  \titlepage
\end{frame}


\begin{frame}[fragile]
  \frametitle{ssh}
  \begin{itemize}
  \item For the next few lessons we would be connecting to Hadoop cluster via {\color{mycolorcli} ssh}
  \item You might want to review section 3 from my ``Introduction to Linux and RCC'' tutorial
    {\color{mycolorcli}
\begin{verbatim}
https://canvas.uchicago.edu/courses/23040
\end{verbatim}
    }
    where this is described in more details.
    If you do not have an access to the tutorial, ask at the office. It is a good idea to refresh your knowledge of Linux and review the whole tutorial since the lack of Linux skills is a continuous source of problems.
  \item You would need to have ssh client on your laptop.
    \begin{itemize}
    \item Mac and Linux users already have it by default
    \item Windows users might have to install, for example, MobaXterm
    {\color{mycolorcli}
\begin{verbatim}
https://mobaxterm.mobatek.net/
\end{verbatim}
    }
    \end{itemize}
    
  \item Make sure to enroll in 2-factor authentication:
    {\color{mycolorcli}
\begin{verbatim}
https://2fa.rcc.uchicago.edu/
\end{verbatim}
    }
  \end{itemize}
\end{frame}


\begin{frame}[fragile]
  \frametitle{ssh}
  \begin{itemize}
    \item You need to be either on campus network or use VPN to be able to connect to Hadoop cluster.
      VPN is managed by the university IT and can be downloaded here:
      {\color{mycolorcli}
        { \tiny
\begin{verbatim}
https://uchicago.service-now.com/it?id=kb_article&sys_id=8d7e1ed0db91c0d0432f7f8cbf96195f
\end{verbatim}
        }
      }
  \item Notice: you cannot use ThinLinc to connect to Hadoop cluster, it only works with midway
  \item To ssh to Hadoop cluster:
    \begin{itemize}
    \item Connect to VPN
    \item Open a terminal and type there:
      {\color{mycolorcli}
\begin{verbatim}
ssh YourUserName@hadoop.rcc.uchicago.edu
\end{verbatim}
      }
    \item You'll be prompted for password and 2-factor authentication
    \item Notice: if you mistype your password 3 times, you would be locked out of your account for 4 hours
    \end{itemize}
  \end{itemize}
\end{frame}

\end{document}
